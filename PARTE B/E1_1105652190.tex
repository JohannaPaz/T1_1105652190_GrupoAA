\documentclass[10pt]{report}
\usepackage[spanish]{babel}
\usepackage[utf8]{inputenc}
\usepackage{amsmath,amsthm,amsfonts,amssymb}
\usepackage{Sweave}
\usepackage{graphicx}
\usepackage{hyperref}
\usepackage{anysize} 
\usepackage{lscape}
\usepackage{multirow} % para las tablas
\marginsize{1.5cm}{1.5cm}{0.5cm}{0.5cm} 

\title{\Huge Universidad Nacional de Loja \\ 
Área de la Energía las Industrias y los Recursos Naturales no Renovables \\
Ingeniería en Sistemas \\}

\author{\includegraphics[width=6cm, height=6cm]{unloja.png}\\\\   
  JOHANNA ESTEFANIA PAZ JIMENEZ \\ ECINF7330\\  \texttt{johanna.paz@unl.edu.ec}\\ \\
}

\begin{document}
\Sconcordance{concordance:E1_1105652190.tex:E1_1105652190.Rnw:%
1 62 1 1 2 38 0 1 2 17 1}

\maketitle
\begin{center}\textbf{\Large TITANIC}\end{center}

\textbf{Descripción}\\

Este conjunto de datos proporciona información sobre el destino de los pasajeros en el primer viaje fatal del trasatlántico Titanic, que se resumen de acuerdo a la situación económica (clase), el sexo, la edad y la supervivencia.\\

\textbf{Uso}\\

Titánico.\\

\textbf{Formato}\\

Una matriz de 4-dimensional resultante de cruzada tabular 2201 observaciones sobre 4 variables. Las variables y sus niveles son los siguientes:

No	Nombre	Niveles
1	Clase	Primero, segundo, tercero, Tripulación
2	Sexo	Hombre, Mujer
3	Años	Niño, Adulto
4	Sobrevivieron	No si\\


\textbf{Detalles}

El hundimiento del Titanic es un evento famoso, y nuevos libros siguen siendo publicado sobre el tema. Muchos hechos-de conocidas las proporciones de los pasajeros de primera clase a la política de "mujeres y niños primero ', y el hecho de que esa política no era un éxito completo en el ahorro de las mujeres y niños en la tercera clase se reflejan en la supervivencia tarifas de diversas clases de pasajeros.

Estos datos fueron recogidos originalmente por la Junta Británica de Comercio en su investigación del hundimiento. Tenga en cuenta que no hay un acuerdo completo entre las fuentes primarias como a las cifras exactas a bordo, rescatados, o perdidos.

Debido, en particular, a la película de gran éxito 'Titanic', los últimos años vieron un aumento en el interés público en el Titanic. Datos muy detallados sobre los pasajeros ya está disponible en Internet, en sitios como la Enciclopedia Titanica (http://www.rmplc.co.uk/eduweb/sites/phind).\\

\textbf{Fuente}\\

Dawson, Robert J. MACG. (1995), El 'Episodio inusual' Datos Revisited. Diario de Estadísticas de Educación, 3. Http://www.amstat.org/publications/jse/v3n3/datasets.dawson.html

La fuente proporciona un conjunto de datos de clase de grabación, el sexo, la edad y el estado de supervivencia para cada persona a bordo del Titanic, y se basa en datos recogidos originalmente por la Junta Británica de Comercio y reimpresos en:

Junta Británica de Comercio (1990), Informe sobre la pérdida del 'Titanic' (SS). Junta Británica de Comercio Informe Investigación (reimpresión). Gloucester, Reino Unido: Allan Sutton Publishing.\\

%\textbf{}\\

\textbf{2. Tabla Titanic:}\\
\begin{Schunk}
\begin{Soutput}
, , Age = Child, Survived = No

      Sex
Class  Male Female
  1st     0      0
  2nd     0      0
  3rd    35     17
  Crew    0      0

, , Age = Adult, Survived = No

      Sex
Class  Male Female
  1st   118      4
  2nd   154     13
  3rd   387     89
  Crew  670      3

, , Age = Child, Survived = Yes

      Sex
Class  Male Female
  1st     5      1
  2nd    11     13
  3rd    13     14
  Crew    0      0

, , Age = Adult, Survived = Yes

      Sex
Class  Male Female
  1st    57    140
  2nd    14     80
  3rd    75     76
  Crew  192     20
\end{Soutput}
\end{Schunk}
\textbf{3. ¿Es aplicable la ingeniería de software cuando se elaboran webapps? Si es así, ¿cómo puede
modificarse para que asimile las características únicas de éstas?}\\

RESPUESTA:Si es aplicable xq permite brindar alos ingenieros brindar una buena capacidad de computo
\\


\textbf{4. Numero de casos del dataset:}\\
El numero de caso es 2201\\
sum(Titanic)\\




\begin{center}

\end{center}
\end{document}
